\documentclass{article}
\usepackage[margin=1in]{geometry}
\usepackage{enumitem}
\usepackage{hyperref}

\usepackage{array}
\newcolumntype{R}[1]{>{\raggedleft}m{#1}}

\begin{document}

\begin{center}
    \textbf{\Large Zifan Hua}
\end{center}

\begin{center}
    Tel: +44 787 367 6711 Email: \href{mailto:huazifan@gmail.com}{huazifan@gmail.com}
\end{center}

\section*{Education}
    \vspace{-0.7em}
    \hrule
    \vspace{0.3em}

    \noindent
    \begin{tabular}{
        @{} p{0.70\textwidth}
        @{} R{0.30\textwidth} ll@{}
    }
        \textbf{University of Nottingham}
        & \textbf{Nottingham} & \\
        BSc Hons Maths, GPA: 91.33/100
        & 2023/Sep - 2025/Jun
    \end{tabular}
    \noindent
    \textbf{Core Courses:}
    Algebra,
    Real Analysis,
    Complex Analysis,
    Number Theory

    \noindent
    \begin{tabular}{
        @{} p{0.70\textwidth}
        @{} R{0.30\textwidth} ll@{}
    }
        \textbf{University of Nottingham Ningbo China}
        & \textbf{Ningbo} & \\
        BSc Hons Maths w/App Maths,
        GPA: 81.42/100 (First Year), 86.92/100 (Second Year)
        & 2021/Sep - 2023/Jun
    \end{tabular}
    \noindent
    \textbf{Core Courses:}
    Statistics, Python, Applied Mathematics, Probability, Calculus

    \noindent
    \textbf{Awards:}
    Zhejiang Provincial Government Scholarship,
    First Prize in International Mathematical Competition (2023/Aug),
    Second Prize in International Mathematical Competition (2022/Aug),
    First Prize in UNNC Mathematical Competition (2022/May),
    First Prize in UNNC Mathematical Competition (2023/May)

\section*{Professional Experience}
    \vspace{-0.7em}
    \hrule
    \vspace{0.3em}

    \noindent
    \begin{tabular}{
        @{} p{0.80\textwidth}
        @{} R{0.20\textwidth} ll@{}
    }
        \textbf{Weekly Undergraduate Seminar in Mathematics: Communication Without Error}
        & \textbf{Ningbo} & \\
        Participant and Speaker
        & 2022/Sep - 2023/Jun
    \end{tabular}

    \noindent
    \textbf{Supervisor:} Dr. Hamid Reza Daneshpajouh

    \noindent
    \begin{itemize}[topsep=0.0em,itemsep=0.0em,parsep=0.0em]
        \item Given two talks about the topic.
        \item Investigate the definition and methods of computation of Shannon Capacity.
        \item Investigate the method used to calculate the Shannon capacity of cyccle graph $C_5$ introduced by László Lovász that linked linear algebra and combinatory
        \item Discuss the latest result and open problems in this area.
    \end{itemize}

    \noindent
    \begin{tabular}{
        @{} p{0.80\textwidth}
        @{} R{0.20\textwidth} ll@{}
    }
        \textbf{Classicalisation of Generalised Swiss Cheeses in Banach spaces}
        & \textbf{Nottingham} & \\
        Participant
        & 2024/Jun - 2023/Aug
    \end{tabular}
    \textbf{Supervisor:} Dr. Joel Feinstein
    \begin{itemize}[topsep=0.0em,itemsep=0.0em,parsep=0.0em]
        \item Defining and investigate different kind of non-standard Swiss Cheese such as Cheese with convex bubbles and general open sets, and measuring the Cheese with different ways of measurement such as diameter, perimeter, inner and outer radius.
        \item Proving the result that $R(X) \neq C(X)$ on Swiss Cheese with convex bubbles using Tietze extension theorem and direct prove.
        \item Proving the classicalisation Theorem of Generalised Swiss Cheeses using transfinite induction and semi-classicalisation with direct induction.
        \item Investigate the generality of the coalescence process used in the proof of classicalisation theorem, and formalize the process by a theorem. Then compare the theorem with Zorn's lemma.
    \end{itemize}

\section*{Skills}
    \vspace{-0.7em}
    \hrule
    \vspace{0.3em}

    \begin{itemize}[topsep=0.0em,itemsep=0.0em,parsep=0.0em]
        \item \textbf{Languages:} English, Chinese
        \item Proficient in C/C++, Python, Rust, \LaTeX
    \end{itemize}

\end{document}